\documentclass[12pt]{article}
\usepackage{fontspec, xeCJK}

\setmainfont{Times New Roman}
\setsansfont{Comic Sans MS}
\setmonofont{Source Code Pro}
\setCJKmainfont{Songti SC}
\setCJKsansfont{PingFang SC}
\setCJKmonofont{Inziu Iosevka SC}

\usepackage[a4paper]{geometry}
\setlength\parindent{0pt}
\setlength{\parindent}{2em} % 首行缩进
\usepackage{indentfirst} % 确保首行缩进
\linespread{1.5} % 行距

\usepackage{xcolor}
\usepackage{listings} %插入代码用
\lstset{
    basicstyle={\small\ttfamily}, 
    commentstyle=\color{red!50!green!50!blue!50}, 
    numbers=left, 
    numberstyle={\small\ttfamily\color{red!50!green!50!blue!50}}, 
    frame=shadowbox, 
    rulesepcolor= \color{ red!50!green!50!blue!50}, 
    xleftmargin=2em,
    xrightmargin=2em, 
    aboveskip=1em
}
\usepackage{hyperref} %添加目录链接
\usepackage{amsmath} % 数学符号
\usepackage{amssymb}

\title{数据结构与算法定义}
\date{}
\begin{document}
\maketitle
\tableofcontents

\section{线性表定义}

将线性表表示为 $$(a_1, a_2, ..., a_{i-1}, a_i, a_{i+1}, ..., a_n)$$ 
其中 $a_i$ 为表中元素,元素的类型需一致。表中 $a_{i-1}$ 领先于 $a_i$ ,$a_i$ 领先于 
$a_{i+1}$ ,则称 $a_{i-1}$ 是 $a_i$ 的\textbf{直接前驱},$a_{i+1}$ 是 $a_i$ 的
\textbf{直接后继}。每个元素都有确定的位置,如 $a_1$ 是第一个元素;
$a_i$ 中的 $i$ 是该元素的\textbf{位序}。

表内的元素个数 $n(n\geq0)$ 为线性表的\textbf{长度},$n=0$ 时,线性表为空表。

\section{栈定义}

栈 (Stack) 是限定仅在表尾进行插入或删除操作(操作受限)的线性表。
不含元素的空表称为空栈。

将栈表示为 $$(a_1, a_2, ..., a_{i-1}, a_i, a_{i+1}, ..., a_n)$$ 
其中 $a_1$ (表头)是栈底,$a_n$ (表尾)是栈顶。对栈的修改应只在栈顶元素处进行。

栈内元素遵循后进先出原则,类似羽毛球筒,或汉诺塔。

\section{队列定义}

队列 (Queue) 是遵循先进先出原则的操作同样受限的线性表,只允许在表的一端插入,另一端删除元素。允许插入的一端称为\textbf{队头 (front)} ,允许删除的一端称为\textbf{队尾 (rear) }。

\end{document}
